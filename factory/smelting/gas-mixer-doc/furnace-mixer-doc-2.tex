% arara: lualatex: {shell: 1}
\documentclass{article}
\usepackage{geometry}[margin=1in]
\usepackage[T1]{fontenc}
\usepackage{booktabs}
\usepackage[inline]{enumitem}
\usepackage[colorlinks,unicode]{hyperref}
\usepackage{subcaption}
\usepackage{minted}
\usepackage{multicol}
\usepackage{paracol}
\usepackage{mathpazo}
\usepackage{amsmath}
\usepackage{amssymb}
\usepackage{mathtools}
\usepackage{xfrac}
\usepackage{pgfplots}
\pgfplotsset{compat=1.17}
\usepgflibrary{fpu}

\def\lexer{stationeers-mips-lexer-dumb.py:CustomLexer -x}
\newminted[mips]{\lexer}{autogobble,escapeinside=||}

\newcommand\pgfmathparseFPU[1]{
    \begingroup%
        \pgfkeys{/pgf/fpu,/pgf/fpu/output format=fixed}
        \pgfmathparse{#1}
        \pgfmathsmuggle\pgfmathresult
    \endgroup
}

\newcommand{\tightmathT}{%
    \setlength{\abovedisplayskip}{0pt}
    \setlength{\abovedisplayshortskip}{0pt}
}
\newcommand{\tightmathB}{%
    \setlength{\belowdisplayskip}{0pt}
    \setlength{\belowdisplayshortskip}{0pt}
}
\newcommand{\tightmath}{%
    \tightmathT%
    \tightmathB%
}

\DeclareMathOperator{\CDiox}{\mathrm{CO}_2}
\DeclareMathOperator{\R}{\mathbb{R}}

\begin{document}

\def\GasConstant{8.314}
\def\vF{1000}
\def\tC{230}
\def\tH{2300}

\newcommand{\setupparams}[4]{%
    \pgfmathsetmacro{\pF}{#1}
    \pgfmathsetmacro{\tF}{#2}
    \pgfmathparseFPU{\pF/\tF*\vF/\GasConstant)}\pgfmathsetmacro{\nF}{\pgfmathresult}
    \pgfmathsetmacro{\pT}{#3}
    \pgfmathsetmacro{\tT}{#4}
    \pgfmathparseFPU{\pT/\tT*\vF/\GasConstant)}\pgfmathsetmacro{\nT}{\pgfmathresult}
    \pgfmathsetmacro{\rG}{0}
    \pgfmathparseFPU{\nF-(\nT*(\tT-\tH))/(\tF-\tH)}\pgfmathsetmacro{\rR}{\pgfmathresult}
    \pgfmathparseFPU{\nF-(\nT*(\tT-\tC))/(\tF-\tC)}\pgfmathsetmacro{\rB}{\pgfmathresult}
}
\newcommand{\surf}{%
    \addplot3[%
        surf,
        samples=15,
        samples y=15,
        opacity=0.5,
        domain=0:\nF,
        domain y=\tC:\tH,
        variable=\nR,
    ]
    {(\tT*\nT-\tF*(\nF-\nR))/y};
}
\newcommand{\curve}{%
    \addplot3[%
        thick,
        color=red,
        samples=30,
        samples y=0,
        domain={max(\rG, max(\rR, \rB))}:\nF,
        domain y=\tC:\tH,
        variable=\nR,
    ]
    ({\nR},{(\tT*\nT-\tF*(\nF-\nR))/(\nT-\nF+\nR)},{\nT-\nF+\nR});
}
\newcommand{\Surf}[5][1.0]{%
    \setupparams{#2}{#3}{#4}{#5}
    \begin{tikzpicture}[scale=#1]
        \begin{axis}[
                title={$p_F=\pF$, $t_F=\tF$, $p_T=\pT$, $t_T=\tT$},
                xlabel=$n_R$, ylabel=$t_I$, zlabel=$n_I$,
                view={-135}{45},
                ymin=\tC, ymax=\tH,
                zmin=0,
            ]
            \surf%
            \curve%
        \end{axis}
    \end{tikzpicture}
}
\newcommand{\Full}[5][1.0]{%
    \setupparams{#2}{#3}{#4}{#5}
    \begin{tikzpicture}[scale=#1]
        \begin{axis}[
                title={$p_F=\pF$, $t_F=\tF$, $p_T=\pT$, $t_T=\tT$},
                xlabel=$n_R$, ylabel=$t_I$, zlabel=$n_I$,
                view={-135}{45},
                ymin=\tC, ymax=\tH,
                zmin=0,
            ]
            \surf%
            \curve%
            \filldraw[green]
            (
                {\rG},
                {(\tT*\nT-\tF*\nF)/(\nT-\nF)},
                {\nT-\nF}
            ) circle (2pt) node {};
            \filldraw[red]
            (
                {\rR},
                {\tH},
                {\nT-(\nT*(\tT-\tH))/(\tF-\tH)}
            ) circle (2pt) node {};
            \filldraw[blue]
            (
                {\rB},
                {\tC},
                {\nT-(\nT*(\tT-\tC))/(\tF-\tC)}
            ) circle (2pt) node {};
        \end{axis}
    \end{tikzpicture}
}

\title{Optimal One-gas Model Furnace}
\maketitle

\tableofcontents

\section{Introduction}

A problem that most players face in Stationeers is automating a furnace
for smelting alloys, where the issue is reaching a specific pressure target
and temperature target. With a naïve setup, we combust fuel to raise the
pressure and temperature, cool the system if need be, then remove pressure
until both are in range. A better, more precise setup is demanded for
more difficult alloys.

\section{Motivation}

The immediate question is: given the initial conditions of a furnace,
can I add a specifically formated mixture of gas in order to reach those
targets without guess-work? The answer is yes: though we may need to remove
some gas first, there is such a perfect mixture.
However, embedded within this problem of temperature and pressure is
an issue of specific heat. A volume $M$ comprised of a mix of the seven
gasses in the game may have a particular state (parameters pressure, temperature
and moles per gas), but there are many different gas compositions such that
the specific heat $s_M$ will satisfy these conditions.
In-fact, all such possible compositions lie within an $\R^7$ subspace, whose
support is on what gasses exist in the volume.
\begin{gather*}
    s_M
    = \vec{n_M}\cdot\vec c
    = (n_M(g_0),\cdots,n_M(g_6))\cdot(c_{g_0},\cdots,c_{g_6}) \\
    \vec{n_M}_g = n_M(g). \tag{\text{The $g^{\textrm{th}}$ component}}
\end{gather*}
The dimension of this subspace depends on the number of gasses in the system,
thus intentionally restricting the gasses (as per filtering) allows us to reduce
the dimension as we see fit.
This, along with the practical purposes of only using one gas (namely, removing
gases which have useful applications outside pressurizing a volume)
is why I've decided to implement a one-gas model furnace.
In such a system, the dimension of the specific heat subspace is reduced to
a single dimension, and calculation of the energy from combining two
compositions is made significantly simpler.
If we fix $g^\ast$ the sole gas of the system, then
the specific heat $s_M$ of a volume $M$ becomes:
\[
    n_M = \sum_{g\in G}n_M(g) = n_M(g^\ast),\quad
    \vec{n_M}_g = \begin{cases}
        n_M, & \text{if }g=g^\ast \\
        0, &\text{if }g\ne g^\ast
    \end{cases}\\
    \Rightarrow
    s_M
    = \vec{n_M}\cdot\vec{c}
    = n_M c_{g^\ast}.
\]
Furthermore, equation defining the result of combining two volumes with
different temperatures becomes significantly simpler, where specific heat
is replaced simply by moles:
\[
    t_C s_C = t_A s_A+t_B t_B\
    \Rightarrow\
    t_C n_C c_{g^\ast} = t_A n_A c_{g^\ast}+t_B n_B c_{g^\ast}\
    \Rightarrow\
    t_C n_C = t_A n_A+t_B n_B.
\]

\section{Implementation and optimization}

Now we move into the specifics of implementing such a system in the game,
involving furnaces and pipe networks.
Considering a system of only one gas with
a hot source $H$ of temperature $t_H$ and
a cold source $C$ of temperature $t_C$,
we can calculate an optimal formulation for bringing a furnace $F$
(with volume $v_F$, initial pressure $p_F$, initial temperature $t_F$,
and initial moles $n_F=\frac{p_F v_F}{R t_T}$, where
$R$ is the \href{https://en.wikipedia.org/wiki/Gas_constant}{ideal gas constant})
to a desired pressure $p_T$ and temperature $t_T$.
This is accomplished by removing an amount $n_R$ from the furnace and/or adding
an amount $n_I$ at a specific temperature $t_I$, where $I$ is composed from
amounts $n_H$ and $n_C$ from the $H$ and $C$ sources.
Several gas-law derived equations constrain this process:
\begin{gather}
    t_T n_T = t_F(n_F-n_R)+t_I n_I \label{eq:surface} \\
    n_T = n_F-n_R+n_I \label{eq:constraint} \\
    t_I n_I = t_H n_H+t_C n_C \label{eq:added}
\end{gather}
Where $n_R$, $t_I$ and $n_I=n_H+n_C$ are to be determined.
Note that each term in this system obeys some constraints:
\begin{gather*}
    t_C<t_M<t_H,\quad
    0\le n_M,\quad
    0\le n_R\le n_F.
\end{gather*}
Solving \autoref{eq:surface} for $n_I$ provides
a surface bounded in two dimensions
by $0\le n_R\le n_F$ and $t_C\le t_I\le t_H$,
but where $0\le n_I$ is potentially unbounded.
\[
    (n_R,t_I,f):\quad
    f(n_R,t_I) = n_I = \frac{t_T n_T+t_F(n_R-n_F)}{t_I}.
\]
\autoref{eq:constraint} further restricts potential solutions, but
given a satisfactory amount to remove $n_R$ provides a particular solution
for the amount to add: $n_I=n_T-n_F+n_R$.
Then we instead solve \autoref{eq:surface} for $t_I$,
and as a result these restricted solutions lie
within a curve embedded within the surface.
\[
    (n_R,h,g):\quad
    g(n_R) = n_I = n_T-n_F+n_R,\quad
    h(n_R) = t_I = \frac{t_T n_T-t_F(n_F-n_R)}{n_T-n_F+n_R}.
\]
\begin{figure}
    \begin{center}
        \Surf{4000}{500}{9000}{700}
    \end{center}
    \caption[]{%
        The $(n_R,t_I,f)$ solution surface
        and $(n_R,h,g)$ embedded curve.
    }
    \label{fig:surface}
\end{figure}
\noindent
With respect to $t_I$, $h$ is monotone decreasing,
but monotone increasing with respect to $n_R$.
Thus minimizing with respect to $n_I$ in the domain is
a matter of minimizing $n_R$ and maximizing $t_I$.
However, $0\le n_R$ is naïve, and \autoref{eq:added}
imparts a narrower lower-bound.
\begin{multicols}{2}
    \raggedcolumns
    \tightmath
    \begin{align*}
        t_C n_I &\le t_I n_I \\
        n_F(n_F-t_C)-n_T(t_T-t_C) &\le n_R(t_F-t_C)
    \end{align*}
    \begin{equation}\label{eq:nRC}
        n_{R_C} = n_F-\frac{n_T(t_T-t_C)}{t_F-t_C} \le n_R
    \end{equation}
    \columnbreak

    \begin{align*}
        t_I n_I &\le t_H n_I \\
        t_Tn_T-t_F(n_F-n_R) &\le t_H(n_T-n_F+n_R)
    \end{align*}
    \begin{equation}\label{eq:nRH}
        n_{R_H} = n_F-\frac{n_T(t_H-t_T)}{t_H-t_F} \le n_R
    \end{equation}
\end{multicols}
\noindent
Thus $\max(n_{R_C},n_{R_H}) \le n_R$.
In-fact, the curve crosses a hyperplane boundary at a point $(n_R,t_I,n_I)$
which minimizes $n_I$ and for which $n_R=\max(0, n_{R_C}, n_{R_H})$.
We then easily solve for the remaining coordinates.

In the end, finding the $n_I$ minimizing point $(n_R,t_I,n_I)$:
\begin{align}
    n_R &= \max(0, n_{R_C}, n_{R_H}) \label{eq:nR} \\
    t_I &= \frac{t_T n_T-t_F(n_F-n_R)}{n_T-n_F+n_R} \label{eq:tI} \\
    n_I &= n_T-n_F+n_R. \label{eq:nI}
\end{align}
Which hyperplane the curve crosses corresponds to a certain procedure of
the gas mixer that in the end we are making:
the curve either:
\begin{itemize}
    \item Crosses $n_R=0$:
        a mixture of $H$ and $C$ is added, or
    \item Crosses $t_I=t_H$:
        some volume of $F$ is removed and only $H$ is added, or
    \item Crosses $t_I=t_C$:
        some volume of $F$ volume is removed and only $C$ is added.
\end{itemize}
\begin{figure}
    \begin{center}
        \begin{subfigure}{0.48\textwidth}
            \Full[.7]{6000}{600}{21000}{900} % green example
            \caption{$H$ and $C$ mixture added.}
        \end{subfigure}
        \vspace{1em}

        \begin{subfigure}{0.48\textwidth}
            \Full[.7]{6000}{600}{5000}{800} % red example
            \caption{Volume removed, $H$ added.}
        \end{subfigure}
        \begin{subfigure}{0.48\textwidth}
            \Full[.7]{15000}{900}{6000}{600} % blue example
            \caption{Volume removed, $C$ added.}
        \end{subfigure}
    \end{center}
    \caption{%
        Example solutions.
    }
\end{figure}
Lastly, since $n_I$ is composed of moles from $H$ and $C$,
we also need to calculate $n_H$ and $n_C$,
which is trivial given that $n_I=n_H+n_C$ and $t_I n_I=t_H n_H+t_C n_C$.
\begin{equation}\label{eq:nC}
    \begin{aligned}[b]
        t_I &= n_H+n_C \\
        \Rightarrow n_C &= n_I-n_H.
    \end{aligned}
\end{equation}
\begin{equation}\label{eq:nH}
    \begin{aligned}[b]
        t_I n_I
        &= t_H n_H+t_C n_C \\
        &= t_H n_H+t_C(n_I-n_H) \\
        &= t_C n_I+n_H(t_H-t_C) \\
        \Rightarrow n_H &= \frac{t_I n_I-t_C n_I}{t_H-t_C}.
    \end{aligned}
\end{equation}
And what we are left with is precisely the minimum number of moles $n_R$ to
remove from the furnace initially, and (with respect to $n_R$) $n_H$ and $n_C$
the precise numbers of moles to add from the $H$ and $C$ sources respectively,
such that $F$ will achieve pressure $p_T$ and temperature $t_T$ exactly.
All that is left is game implementation.

\section{Game implementation with MIPS}

I've decided to implement this entire mixing program along with controls onto a
single IC housing, and implementation details will depend on what devices we
specifically need.

Note that for or clarity in describing the steps of the MIPS program,
many uniquely labeled term of an equation will have a unique register alias
in the code described below, but optimizations will need to be made in
reusing registers in order to stay within the 16 register limit of the
final MIPS program.
\vspace{1em}
\begin{paracol}{2}
    \noindent
    I've chosen to use the following device inputs:
    the furnace itself, for reading its state,
    three pipe analyzers; one each for $H$, $C$ and $I$,
    one volume pump for removing the $n_R$ initial furnace moles, and
    one gas mixer for composing $I$.
    \switchcolumn
    \vspace{-1em}
    \begin{mips}
        alias HAnalyzer d0
        alias HPump d1
        alias CAnalyzer d2
        alias CPump d3
        alias IAnalyzer d4
        alias IPump d5

        alias x <num> # for calulcations
        alias y <num>
        alias z <num>
        alias v <num> # for volume in calculations
        alias i <num> # for device indirection
    \end{mips}
\end{paracol}

\subsection{Constants and utility functions}

\begin{paracol}{2}
    Define the ideal gas constant and volumes.
    \switchcolumn
    \begin{mips}
        define R 8.31446261815324
        define vF  1000
        define vHC 500  #  5*100L per pipe
        define vI  1500 # 15*100L per pipe
    \end{mips}
    \switchcolumn*
    \noindent
    For emptying the input pipe $I$ into the furnace
    \switchcolumn
    \begin{mips}
        fillFurnace:
        yield
        s Furnace SettingInput 100
        l x IAnalyzer TotalMoles
        brgtz -3
        s Furnace SettingInput 0
        j ra
    \end{mips}
    \switchcolumn*
    \noindent
    And for emptying the furnace into the filtration system,
    which filters all $\CDiox$ back into the input pipe.
    \switchcolumn
    \vspace{-1em}
    \begin{mips}
        emptyFurnace:
        yield
        s Furnace SettingOutput 100
        l x Furnace TotalMoles
        brgtz -3
        s Furnace SettingOutput 0
        l x IAnalyzer TotalMoles
        yield
        l y IAnalyzer TotalMoles
        beq x y ra # wait for CO2 filter
        move x y
        jr -4
    \end{mips}
\end{paracol}

\subsection{%
    Acquiring \texorpdfstring{$p_T$}{pT} and \texorpdfstring{$t_T$}{tT}
}

Since all six of our device pins are being used already, we have no way
to directly set the target pressure and temperature for our program,
or even any way to directly start it.
Instead we can use a messaging system, where another program (a control program)
sends messages which set these values or start the program.
The \verb+db+ register (that is, the value corresponding to the IC housing)
acts as a channel through which a ``sender'' can communicate to this one,
the ``receiver'', and vice versa.
\begin{center}
    \begin{tabular}{rl}
        \toprule
        \verb+db+ value & Description \\
        \midrule
        \verb+0+        & Receiver (this program) is ready to receive a message \\
        \verb+-1+       & Receiver awaiting a follow-up value \\
        \verb+1+        & Target pressure message \\
        \verb+2+        & Target temperature message \\
        \verb+3+        & Start message \\
        \bottomrule
    \end{tabular}
\end{center}
\begin{paracol}{2}
    First we define the messages.
    \switchcolumn
    \begin{mips}
        define MsgReadyMsg 0
        define MsgReadyValue -1
        define MsgPTarget 1
        define MsgTTarget 2
        define MsgStart 3
    \end{mips}
\end{paracol}

The receiver program contains in its main loop idle state
a series of branches based on the kind of message it has received.
In the case that \verb+db+ has been set to \verb+1+/\verb+2+ then
the receiver understand that a sender is going to send a value corresponding
to target pressure/temperature. The receiver sets itself to \verb+-1+ to
announce to any and all senders that were \emph{not} to program that sent
the message to not send anything, and to announce to the actual sender that
this receiver is ready to accept whatever the actual value is. The actual
sender waits to see \verb+-1+, before overwriting it with the actual value.
Lastly the receiver sees the value, moves it into the appropriate register,
and sets itself to \verb+0+ to announce that it is once again ready to
accept messages. In the case that \verb+3+ was sent, then the receiver instead
``starts'', and moves on to the actual gas mixing program.
\vspace{1em}
\begin{paracol}{2}
    \noindent
    The receiver message component.
    In addition to receiving $p_T$ and $t_T$,
    we also need to recalculate the target moles $n_T$.
    \begin{mips}
        checkInput:
        l x db Setting
        brne x MsgPTarget 4 # receive pT
        jal waitReceive
        move pT x
        jal calculateTargetMoles
        brne x MsgTTarget 4 # receive tT
        jal waitReceive
        move tT x
        jal calculateTargetMoles
        bne x MsgStart main
        # ...

        waitReceive:
        s db Setting MsgReadyValue
        yield
        l x db Setting
        breq x MsgReadyValue -2
        s db Setting MsgReadyMsg
        j ra

        calculateTargetMoles:
        mul nT pT FurnaceVolume # nT ...
        div nT nT R
        div nT nT tT # ... = (pT*vT)/(R*tT)
        j ra
    \end{mips}
    \switchcolumn
    \noindent
    Example sender message component
    \begin{mips}
        checkInput:
        l x GasControlIC Setting
        bne x MsgReadyMsg ra # skip if rx busy
        l x PTargetDial Setting # pTarget
        breq x pTarget 6
        move pTarget x
        mul x x 100
        s PTargetDisplay Setting x
        move y MsgPTarget
        j waitSend
        l x TTargetDial Setting # tTarget
        breq x tTarget 6
        move tTarget x
        mul x x 20
        s TTargetDisplay Setting x
        move y MsgTTarget
        j waitSend
        l x StartButton Activate # start
        breq x start 5
        move start x
        breqz start 3 # skip if changed to 0
        move y MsgStart
        s GasControlIC Setting y
        j ra

        waitSend:
        s GasControlIC Setting y
        s db Setting 1
        yield
        l y GasControlIC Setting
        brne y MsgReadyValue -2
        s GasControlIC Setting x
        s db Setting 0
        j ra
    \end{mips}
\end{paracol}

\subsection{%
    Calculating \texorpdfstring{$n_R$}{nR}, \texorpdfstring{$t_I$}{tI},
    \texorpdfstring{$n_I$}{nI}, and \texorpdfstring{$n_H$}{nH}
}

Once the start message has been sent, we carry out the entire
gas mixing procedure for the current state of $F$, $H$ and $C$.
\vspace{1em}
\begin{paracol}{2}
    \noindent
    \[
        n_{R_C} = n_F-\frac{n_T(t_T-t_C)}{t_F-t_C} \le n_R
    \]
    \switchcolumn
    \begin{mips}
        calculate:
        sub nRC tT tC
        mul nRC nT nRC
        sub x tF tC
        div nRC nRC x
        sub nRC nF nRC # nRC=nF-nT(tT-tC)/(tF-tC) (eq.|\ref{eq:nRC}|)
    \end{mips}
    \switchcolumn*
    \noindent
    \[
        n_{R_H} = n_F-\frac{n_T(t_H-t_T)}{t_H-t_F} \le n_R
    \]
    \switchcolumn
    \begin{mips}
        sub nRH tT tH
        mul nRH nT nRH
        sub x tH tF
        div nRH nRH x
        sub nRH nF nRH # nRH=nF-nT(tH-tT)/(tH-tF) (eq.|\ref{eq:nRH}|)
    \end{mips}
    \switchcolumn*
    \noindent
    \[
        n_R = \max(n_{R_C},n_{R_H},0)
    \]
    \switchcolumn
    \begin{mips}
        max nR nRC nRH
        max nR nR 0 # nR=max(nRC,nRH,0) (eq.|\ref{eq:nR}|)
    \end{mips}
    \switchcolumn*
    \noindent
    \[
        n_I = n_T-n_F+n_R
    \]
    \switchcolumn
    \begin{mips}
        sub nI nT nF
        add nI nI nR # nI=nT-nF+nR (eq.|\ref{eq:nI}|)
    \end{mips}
    \switchcolumn*
    \noindent
    \[
        t_I = \frac{t_T n_T-t_F(n_F-n_R)}{n_I}
    \]
    \switchcolumn
    \begin{mips}
        mul tI tT nT
        sub x nF nR
        mul x tF x
        div tI tI nI # tI=(tT*nT-tF(nF-nR))/nI (eq.|\ref{eq:tI}|)
    \end{mips}
    \switchcolumn*
    \noindent
    \[
        n_H = \frac{t_I n_I-t_C n_I}{t_H-t_C}
    \]
    \switchcolumn
    \begin{mips}
        mul nH tI nI
        mul x tC nI
        sub nH nH x
        sub x tH tC
        div nH nH x # nH=(tI*nI-tC*nI)/(tH-tC) (eq.|\ref{eq:nH}|)
    \end{mips}
    \switchcolumn*
    \noindent
    \[
        n_C = n_I-n_H
    \]
    \switchcolumn
    \begin{mips}
        sub nC tI nH (eq.|\ref{eq:nC}|)
    \end{mips}
\end{paracol}

\subsection{%
    Removing \texorpdfstring{$n_R$}{nR},
    mixing, and adding \texorpdfstring{$I$}{I}
}

\hbadness=20000 % Paracol badness is okay

\vspace{1em}
\begin{paracol}{2}
    If $n_R>0$, we empty the furnce into $I$ and use the dump pump to remove
    $n_R$ moles. Using the $I$ pipe analyzers, we calculate $n_{I_0}-n_R$
    (where in this context $n_{I_0}$ is the moles of furnace gas now in $I$
    initially, and $n_I$ at a discrete time)
    and can write the dump pump setting optimally every tick as to not remove
    too much, given:
    \begin{equation}\label{eq:R}
        R\text{ (setting)}=v_I\left(1-\frac{n_I}{n_{I_0}}\right).
    \end{equation}
    Note that the initial moles $n_{M_0}$ of a volume $M$ is assumed to be
    greater than what is needed in the mix $n_M$. If this isn't true, then the
    following code with become stuck in an infinite loop!
    \switchcolumn
    \begin{mips}
        beqz nR mix # skip if nothing to remove
        dump:
        jal emptyFurnace
        l y IAnalyzer TotalMoles
        sub y y nR # target moles I
        dumpLoop:
        yield
        move i d(IAnalyzer)
        move v vI
        jal setPump
        bgtz x dumpLoop # loop if setting > 0
        s IPump On 0
        # ...
        setPump: # y=target, v=volume
        l x dr(i) TotalMoles # discrete moles
        div x y x
        sub x 1 x
        mul x v x # x=(r=v(1-nt/n)) # (eq.|\ref{eq:R}|)
        add i i 1 # move to pump
        s dr(i) Setting x
        s dr(i) On 1
        add i i 1 # move to next analyzer (if any)
    \end{mips}
\end{paracol}
The mixing algorithm diverges depending on whether we chose to use a single
gas mixer, or two volume pumps. One gas mixer is simple, but both less accurate
and much slower than using two simultaneous volume pumps in the same manner as
what we just performed for the furnace gas dumping algorithm.
\vspace{1em}
\begin{paracol}{2}
    \noindent
    (One gas mixer) \\

    \noindent
    With $H$ as input one, the mixer ratio is given by a fraction $R_H$
    in terms of $n_H$, $n_C$, $n_{H_0}$ and $n_{C_0}$,
    where $n_{M_0}$ is the initial amount of gas in pipe network $M$.
    \begin{equation}\label{eq:RH}
        R_H = \frac{r_H}{r_H+r_C},\quad
        r_M = \frac{n_M}{n_{M_0}}.
    \end{equation}
    \begin{mips}
        mix:
        l x HAnalyzer TotalMoles
        l y CAnalyzer TotalMoles
        div x nH x # x=(rH=nH / total H moles)
        div y nC y # y=(rC=nC / total C moles)
        add y x y # y=rH+rC
        div x x y # x=rH/(rH+rC) # (eq.|\ref{eq:RH}|)
        s Mixer Setting x
        yield
        s Mixer On 1
        l x IAnalyzer TotalMoles
        brlt x nI -3
        s Mixer On 0
    \end{mips}
    \switchcolumn
    \noindent
    (Two volume pumps) \\

    \noindent
    We use the same algorithm as dumping $n_R$, but simultaneously with
    sources $H$ and $C$.
    \begin{mips}
        mix:
        l x HAnalyzer TotalMoles
        sub nH x nH # target moles H
        l x CAnalyzer TotalMoles
        sub nC x nC # target moles C
        mixLoop:
        yield
        move i d(HAnalyzer)
        move v vH
        move y nH
        jal setPump
        sgtz z # H setting > 0
        move v vC
        move y nC
        jal setPump
        sgtz x # C setting > 0
        and x x z # either > 0
        bnez x mixLoop # loop if either > 0
        s HPump On 0
        s CPump On 0
    \end{mips}
\end{paracol}
All that is left is to add the $I$ mixture into the furnace,
and we achieve $p_T$ and $t_T$.
\begin{mips}
    jal fillFurnace
    s db Setting 0 # since db=3 all this time
    j main
\end{mips}

\end{document}

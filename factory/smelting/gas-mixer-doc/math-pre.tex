\usepackage{fontspec}
\usepackage[rgb,dvipsnames]{xcolor}
\usepackage[T1]{fontenc}
\usepackage{mathpazo}
\usepackage[letterpaper,margin=1.5in]{geometry}
\usepackage[inline]{enumitem}
%\usepackage{makecell}
%\usepackage{tabu}
\usepackage{multirow}
%\usepackage{multicol}
\usepackage{paracol}
%\usepackage{arydshln}
%\usepackage{subcaption}
\usepackage{booktabs}
\usepackage{array}   % for \newcolumntype macro
\usepackage{hyperref}
\usepackage[backend=biber,style=alphabetic,sorting=ynt]{biblatex}
%\usepackage{float}
%\usepackage{endfloat}
%\usepackage[nomarkers,figuresonly]{endfloat}
%\usepackage[nofiglist,nomarkers]{endfloat}
\usepackage{adjustbox}
\usepackage{caption}
%\usepackage[export]{adjustbox}[2011/08/13]
%\usepackage{graphicx}
\usepackage{amsmath}
\usepackage{amssymb}
\usepackage{amsthm}
\usepackage{mathrsfs}
\usepackage{mathtools}
\usepackage{thmtools}
%\usepackage{calc}
\usepackage{xfrac}
\usepackage[nameinlink]{cleveref}
\usepackage{tikz}
\usetikzlibrary{calc,patterns,decorations.markings,arrows.meta}
\usepackage{pgfplots}
%\usepackage{nth}
\usepackage{cancel}
%\usepackage{comment}
\usepackage{clrscode}
\usepackage{fancyvrb}
\usepackage{minted}

% Minted/code block setup
\setmonofont[Scale=MatchLowercase,Contextuals={Alternate}]{FuraCode Nerd Font}
%\setminted{frame=lines,framesep=1em,fontsize=\small}
\usemintedstyle{gruvbox-light}
\setminted{autogobble,breaklines}

% My colors
\definecolor{colorA}{HTML}{BF4040}
\definecolor{colorB}{HTML}{BF8040}
\definecolor{colorC}{HTML}{4080BF}
\definecolor{colorD}{HTML}{40BFBF}

% Hyperref setup
\hypersetup{
    colorlinks=true,
    linkcolor=colorC,
    urlcolor=colorB,
    citecolor=colorA,
}

%\newcolumntype{C}{>{\begin{math}}c<{\end{math}}}
%\DeclareMathOperator{\SPN}{\text{SPN}}

% Colored left-bar
\newlength{\leftbarwidth}
\setlength{\leftbarwidth}{3pt}
\newlength{\leftbarsep}
\setlength{\leftbarsep}{10pt}
\newcommand*{\leftbarcolorcmd}{\color{leftbarcolor}}
\colorlet{leftbarcolor}{black}
\renewenvironment{leftbar}[1][black]{%
    \colorlet{leftbarcolor}{#1}
    \def\FrameCommand{{\leftbarcolorcmd{\vrule width \leftbarwidth\relax\hspace {\leftbarsep}}}}%
    \MakeFramed {\advance \hsize -\width \FrameRestore }%
}{%
    \endMakeFramed
}

% Theorem environments
\theoremstyle{plain}
\newtheorem{THM}{Theorem}
\newtheorem*{THM*}{Theorem}
\newtheorem{lemma}{Lemma}
\newtheorem*{lemma*}{Lemma}
\newtheorem*{claim}{Claim}
\newtheorem*{cor}{Corollary}
\newtheorem{prop*}{Proposition}
\newtheorem*{conj}{Conjecture}
\theoremstyle{remark}
\newtheorem*{example*}{Example}
\theoremstyle{definition}
\newtheorem*{definition*}{Definition}
\newenvironment{thm}%
  {\begin{leftbar}[colorA]\begin{THM}
}{%
  \end{THM}\end{leftbar}
}
\newenvironment{thm*}%
  {\begin{leftbar}[colorA]\begin{THM*}
}{%
  \end{THM*}\end{leftbar}
}
\newenvironment{prop}%
  {\begin{leftbar}[colorB]\begin{prop*}
}{%
  \end{prop*}\end{leftbar}
}
\newenvironment{definition}%
  {\begin{leftbar}[colorC]\begin{definition*}
}{%
  \end{definition*}\end{leftbar}
}
\newenvironment{example}%
  {\begin{leftbar}[colorD]\begin{example*}
}{%
  \end{example*}\end{leftbar}
}

% Math functions
\newcommand{\spm}[1]{\left(\begin{smallmatrix}#1\end{smallmatrix}\right)}
\newcommand{\gen}[1]{{\langle{#1}\rangle}}
\newcommand{\angvec}[1]{\gen{#1}}
\newcommand{\oldvec}[1]{\vec{#1}}
\renewcommand{\vec}[1]{\boldsymbol{#1}}

% Math operators and symbols
\DeclareMathOperator{\ZZ}{\mathbb{Z}}
\DeclareMathOperator{\QQ}{\mathbb{Q}}
\DeclareMathOperator{\NN}{\mathbb{N}}
\DeclareMathOperator{\RR}{\mathbb{R}}
\DeclareMathOperator{\CC}{\mathbb{C}}
\DeclareMathOperator{\FF}{\mathbb{F}}
\DeclareMathOperator{\calA}{{\mathcal{A}}}
\DeclareMathOperator{\calB}{{\mathcal{B}}}
\DeclareMathOperator{\calC}{{\mathcal{C}}}
\DeclareMathOperator{\calD}{{\mathcal{D}}}
\DeclareMathOperator{\calE}{{\mathcal{E}}}
\DeclareMathOperator{\calF}{{\mathcal{F}}}
\DeclareMathOperator{\calG}{{\mathcal{G}}}
\DeclareMathOperator{\calH}{{\mathcal{H}}}
\DeclareMathOperator{\calI}{{\mathcal{I}}}
\DeclareMathOperator{\calJ}{{\mathcal{J}}}
\DeclareMathOperator{\calK}{{\mathcal{K}}}
\DeclareMathOperator{\calL}{{\mathcal{L}}}
\DeclareMathOperator{\calN}{{\mathcal{N}}}
\DeclareMathOperator{\calP}{{\mathcal{P}}}
\DeclareMathSymbol{\sminus}{\mathbin}{AMSa}{"39}
\DeclareMathOperator{\tr}{tr}
\DeclareMathOperator{\lcm}{lcm}
%\DeclareMathOperator{\Pr}{\mathrm{\textbf{Pr}}}

% Paired math operators
\DeclarePairedDelimiter{\abs}{\lvert}{\rvert}
\DeclarePairedDelimiter{\ceil}{\lceil}{\rceil}
\DeclarePairedDelimiter{\floor}{\lfloor}{\rfloor}

% Math helper functions
\newcommand{\norm}[1]{\lVert{}#1\rVert{}}
\newcommand{\mx}[1]{\begin{matrix}#1\end{matrix}}
\newcommand{\bmx}[1]{\begin{bmatrix}#1\end{bmatrix}}
\newcommand{\pmx}[1]{\begin{pmatrix}#1\end{pmatrix}}
\newcommand{\sbmx}[1]{{\left[\begin{smallmatrix}#1\end{smallmatrix}\right]}}
\newcommand{\spmx}[1]{{\left(\begin{smallmatrix}#1\end{smallmatrix}\right)}}
\newcommand*\circled[3]{%
    \tikz[baseline=(char.base)]{
        \node[circle,draw=#1,inner sep=#2] (char) {#3};
    }
}

% Prompt
%\newif\ifprompts{}
\ifprompts
    \newenvironment{prompt}{\begin{em}\textbf{Prompt:}}{\end{em}}
    %\newenvironment{prompt}{\begin{em}}{\end{em}}
\else
    \excludecomment{prompt}
\fi
\newcommand\pmt[1]{\begin{prompt}#1\end{prompt}}

\pgfplotsset{compat=1.17}

% Style
%\renewcommand\labelitemi{\small$\bullet$}
%\setmonofont[Scale=MatchLowercase,Contextuals={Alternate}]{FuraCode Nerd Font}
%\setminted{frame=lines,framesep=1em,fontsize=\small}
\setlength\parindent{0em}
\setlength\parskip{1em}
%\usemintedstyle{solarizedlight}
%\renewcommand\theadalign{bc}
%\renewcommand\theadfont{\bfseries}
%\renewcommand\theadgape{\Gape[4pt]}
%\renewcommand\cellgape{\Gape[4pt]}
\newcommand\Item{\item\mbox{}}

\makeatletter
\renewenvironment{cases}[1][l]{\matrix@check\cases\env@cases{#1}}{\endarray\right.}
\def\env@cases#1{%
  \let\@ifnextchar\new@ifnextchar
  \left\lbrace\def\arraystretch{1.2}%
  \array{@{}#1@{\quad}l@{}}}
\makeatother

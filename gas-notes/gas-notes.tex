% arara: lualatex: {shell: 1}
\documentclass{article}
\usepackage{booktabs}
\usepackage{hyperref}
\usepackage{minted}
\usepackage{mathpazo}
\usepackage{amsmath}
\usepackage{amssymb}
%\usepackage{amsthm}
%\usepackage{mathrsfs}
\usepackage{mathtools}
%\usepackage{calc}
%\usepackage{xfrac}

\hypersetup{
    colorlinks=true
}

\DeclareMathOperator{\moles}{\mathrm{mol}}
\DeclareMathOperator{\Hyd}{\mathrm{H}_2}
\DeclareMathOperator{\Nit}{\mathrm{N}_2}
\DeclareMathOperator{\Ox}{\mathrm{O}_2}
\DeclareMathOperator{\Pol}{\mathrm{X}}
\DeclareMathOperator{\CDiox}{\mathrm{CO}_2}
\DeclareMathOperator{\Water}{\mathrm{H}_2\mathrm{O}}
\DeclareMathOperator{\NiOx}{\mathrm{N}_2\mathrm{O}}

\title{Stationeers Notes}
\begin{document}

\section{Gas Calculations and MIPS}

The atmospherics simulation in Stationeers currently features seven
gasses that behave according to most ideal gas laws.

\begin{center}
    \begin{tabular}{*3l}
        \toprule
        Name & Formula ($g$) & Specific heat ($c_g$) \\
        \midrule
        Hydrogen & $\Hyd$ & 20.4 \\
        Nitrogen & $\Nit$ & 20.6 \\
        Oxygen & $\Ox$ & 21.1 \\
        Pollutant & $\Pol$ & 24.8 \\
        Carbon dioxide & $\CDiox$ & 28.2 \\
        Water & $\Water$ & 72 \\
        Nitrous oxide & $\NiOx$ & 23 \\
        \bottomrule
    \end{tabular}
\end{center}
Where specific heat is given in joules per mole Kelvin
(and note that a joule is equivalent to a liter kilo-pascal).
We will index these gases via their formula, and let $G$ be the set of indices:
\[
    G = \{\
        \Hyd,\
        \Nit,\
        \Ox,\
        \Pol,\
        \CDiox,\
        \Water,\
        \NiOx\
    \}.
\]

\subsection{Moles and pressures in compositions}

Fix $A$ a gas mixture, and let $g\in G$ be the index of one of the seven gases.
We define $n_A$ to be the total number of moles of any and all gases in $A$,
and $n_A(g)$ to be the number of moles of gas $g$ specifically in $A$.

%For example, if $A=(50\ \moles\ \CDiox,\ 300\ \moles\ \Ox)$, then:
%\begin{itemize}
    %\item $n_A(\CDiox) = 50$,
    %\item $n_A(\Ox) = 300$,
    %\item $n_A = 350$.
%\end{itemize}

Note that $P_{AB}$ is the sum of the pressures (partial pressures)
$P_A$ and $P_B$.

\subsection{Final temperature of mixing two gas compositions}

Let $A$ and $B$ be gas compositions, and let $P_A$, $T_A$ and $n_A(g)$ be the
kilo-pascal pressure, Kelvin temperature, and moles of gas $g\in G$ for
composition $A$; similarly for $B$.
To calculate $T_{AB}$ the final temperature of combining the mixtures in a
one-to-one ratio with a \href{https://stationeers-wiki.com/Pipe_Gas_Mixer}{gas mixer}:
\begin{equation}
    T_{AB}
    = \frac{T_A S_A+T_B S_B}{S_A+S_B}
\end{equation}
Where, for $M_1$ either $A$ or $B$ and
$M_2=A$ if $M_1$ is $B$, or
$M_2=B$ if $M_1$ is $A$
\begin{equation}
    S_{M_1} = p_{M_1}s_{M_1}
\end{equation}
And where
\begin{gather}
    p_{M_1} = \frac{P_{M_2}}{P_{AB}} \\
    s_{M_1} = \sum_{g\in G}s_{M_1}(g) \\
    s_{M_1}(g) = \frac{n_{M_1}(g)}{n_{M_1}}\cdot c_g.
\end{gather}

\inputminted{text}{final-temp.mips}

\end{document}


% arara: lualatex: {shell: 1}
\documentclass{article}
\usepackage{geometry}[margin=1in]
\usepackage{booktabs}
\usepackage{hyperref}
\usepackage{minted}
\usepackage{multicol}
\usepackage{mathpazo}
\usepackage{amsmath}
\usepackage{amssymb}
%\usepackage{amsthm}
\usepackage{mathtools}
\usepackage{xfrac}
%\usepackage{mathrsfs}
%\usepackage{calc}

\hypersetup{
    colorlinks=true
}

\DeclareMathOperator{\moles}{\mathrm{mol}}
\DeclareMathOperator{\Hyd}{\mathrm{H}_2}
\DeclareMathOperator{\Nit}{\mathrm{N}_2}
\DeclareMathOperator{\Ox}{\mathrm{O}_2}
\DeclareMathOperator{\Pol}{\mathrm{X}}
\DeclareMathOperator{\CDiox}{\mathrm{CO}_2}
\DeclareMathOperator{\Water}{\mathrm{H}_2\mathrm{O}}
\DeclareMathOperator{\NiOx}{\mathrm{N}_2\mathrm{O}}

\title{Stationeers Notes}
\begin{document}

\title{Gas Calculations and MIPS}

\maketitle{}

The atmospherics simulation in Stationeers currently features seven
gasses that behave according to most ideal gas laws.

\begin{center}
    \begin{tabular}{*3l}
        \toprule
        Name & Formula ($g$) & Specific heat ($c_g$) \\
        \midrule
        Hydrogen & $\Hyd$ & 20.4 \\
        Nitrogen & $\Nit$ & 20.6 \\
        Oxygen & $\Ox$ & 21.1 \\
        Pollutant & $\Pol$ & 24.8 \\
        Carbon dioxide & $\CDiox$ & 28.2 \\
        Water & $\Water$ & 72 \\
        Nitrous oxide & $\NiOx$ & 23 \\
        \bottomrule
    \end{tabular}
\end{center}
Where specific heat is given in joules per mole Kelvin
(and note that a joule is equivalent to a liter kilo-pascal).
We will index these gases via their formula, and let $G$ be the set of indices:
\[
    G = \{\
        \Hyd,\
        \Nit,\
        \Ox,\
        \Pol,\
        \CDiox,\
        \Water,\
        \NiOx\
    \}.
\]

\section{Moles and pressures in compositions}

Fix $A$ a gas mixture, and let $g\in G$ be the index of one of the seven gases.
We define $n_A$ to be the total number of moles of any and all gases in $A$,
and $n_A(g)$ to be the number of moles of gas $g$ specifically in $A$.

%For example, if $A=(50\ \moles\ \CDiox,\ 300\ \moles\ \Ox)$, then:
%\begin{itemize}
    %\item $n_A(\CDiox) = 50$,
    %\item $n_A(\Ox) = 300$,
    %\item $n_A = 350$.
%\end{itemize}

Note that $P_{AB}$ is the sum of the pressures (partial pressures)
$P_A$ and $P_B$.

\pagebreak
\section{Combining gas compositions}

Let $A$ and $B$ be gas compositions, and let $P_A$ be the kilo-pascal pressure
of composition $A$, $T_A$ the kelvin temperature, $n_A$ the total moles,
and $n_A(g)$ the moles of gas $g$; similarly for $B$.
Then for $T_{AB}$ the resulting temperature of combining the compositions:
\begin{equation}\label{eq:T_AB}
    T_{AB}
    = \frac{T_A S_A+T_B S_B}{S_A+S_B}
\end{equation}
Where, for $M$ either $A$ or $B$
\begin{equation}\label{eq:S_M1}
    S_M = \frac{s_M}{n_M}
\end{equation}
And where
\vspace{-1em}
\begin{multicols}{2}
    \noindent
    \begin{equation}\label{eq:s_M}
        \smash{s_M = \sum_{g\in G}s_M(g)}
    \end{equation}
    \begin{equation}\label{eq:s_M1g}
        \smash{s_M(g) = n_M(g)\cdot c_g}
    \end{equation}
\end{multicols}

\section{Mixing two gas compositions}

\pagebreak
\subsection{Final temperature}

Let $A$ and $B$ be gas compositions, and let $P_A$, $T_A$, $r_A$ and $n_A(g)$ be
the kilo-pascal pressure, Kelvin temperature, mixer ratio, and moles of gas
$g\in G$ for composition $A$; similarly for $B$.
The final temperature $T_{AB}$ of mixing the compositions via either a
\href{https://stationeers-wiki.com/Pipe_Gas_Mixer}{gas mixer}
in a 1:1 ratio, or moving all gas from one valume to another, is calculated:
\begin{equation}\label{eq:T_AB}
    T_{AB}
    = \frac{T_A S_A+T_B S_B}{S_A+S_B}
\end{equation}
Where, for $M_1$ either $A$ or $B$ and
$M_2=A$ if $M_1$ is $B$, or
$M_2=B$ if $M_1$ is $A$
\begin{equation}\label{eq:S_M1}
    S_{M_1} = p_{M_1}r_{M_1}s_{M_1}
\end{equation}
And where
%\setlength{\multicolsep}{6.0pt plus 2.0pt minus 1.5pt}% 50% of original values
\vspace{-1em}
\begin{multicols}{3}
    \noindent
    \begin{equation}\label{eq:p_M1}
        p_{M_1} = \smash{\frac{P_{M_2}}{P_{AB}}}
    \end{equation}
    \begin{equation}\label{eq:s_M1}
        s_{M_1} = \smash{\sum_{g\in G}s_{M_1}(g)}
    \end{equation}
    \begin{equation}\label{eq:s_M1g}
        s_{M_1}(g) = \smash{\frac{n_{M_1}(g)\cdot c_g}{n_{M_1}}}
    \end{equation}
\end{multicols}

\subsection{Total moles in $AB$}

Considering that one composition may have more moles per volume than the other,
the mixer will only function for as long as both compositions have pressure.

\subsection{Final pressure}

Using the ideal gas law,
where $V_{AB}$ is the sum of volumes down downstream from the mixer,
$n_{AB}=n_A+n_B$ the total moles,
and $R\doteq 8.314$ the ideal gas law constant:
\begin{equation}\label{eq:P_AB}
    P_{AB} = \frac{n_{AB}\cdot R\cdot T_{AB}}{V_{AB}}
\end{equation}

%\inputminted{text}{final-temp.mips}

\end{document}

